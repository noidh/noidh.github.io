\documentclass{resume}
\usepackage{xetexko}
\usepackage[utf8]{inputenc}
\usepackage[left=0.75in,top=0.6in,right=0.75in,bottom=0.6in]{geometry}
\usepackage{enumitem}
\usepackage{hyperref}
\hypersetup{
    colorlinks=true,
    linkcolor=blue,
    filecolor=magenta,      
    urlcolor=cyan,
    pdftitle={Doan Huu Noi},
    pdfpagemode=FullScreen,
    }

\urlstyle{same}

\name{DOAN HUU NOI}
\address{ +65-8434-2003 \\ doanhuunoi@gmail.com \\  \url{noidh.github.io}} 

\def\nameskip{\bigskip}
\def\sectionskip{\medskip}

\begin{document}

	\begin{rSection}{Education}
		{\bf Soongsil University, Seoul, South Korea} \hfill {\em Sept 2013 - July 2015} \\  {Master's Degree in Video \& Image Processing.} \smallskip \\
		{\bf Post and Telecom. Institute of Tech., HCM City, Vietnam} \hfill {\em Sept 2008 - Jan 2013} \\  {Bachelor's Degree in Information Technology.}
	\end{rSection}

	\begin{rSection}{Experience}
		\begin{rSubsection}{Zoom}{Nov. 2022 - Current}{Video Processing Software  Engineer}{Singapore}
			\item Optimized core image processing library, accelerating the resampling algorithm with AVX2 intrinsics.
			\item Contributed to 3D telepresence project:\\
			- Conducted the system setup: camera setup,  intrinsic calibration, stereo calibration, multiple cameras synchronization.\\
			- Conducted the data collection: collected synthesis (1,000+ subjects) and real (30 subjects) datasets.\\
			- Improved model performance: implemented a faster half-resolution version, achieved 3× speed-up with TensorRT, and optimized the dataloader.\\
			- Designed and implemented a Python-based demo pipeline (to be ported to C++), with PyTorch and OpenGL Cubemap; accelerated display speed by 10× with GPU tensor rendering.
			\item \underline{Tech Stack:} C++, Python, Pytorch, SIMD, TensorRT, Open3D, OpenCV, OpenGL, Video Processing, Deep Learning, 3D Rendering, 3D Gaussian Splatting,  Depth Estimation,  Model Deployment.

		\end{rSubsection}

		\begin{rSubsection}{MVTech}{Nov. 2018 - Nov.2022}{Image Processing Researcher}{South Korea}
			\item Participated in developing a computer vision processing framework (named RAVID) \\
			- Developed the Shape Finder algorithm, a feature-based matching method capable of effectively handling incomplete or blurry objects. \href{https://blog.naver.com/mvtech_ravid/222119961697}{(View Demo)}\\ 			
- Developing very fast convolution and morphology operations using SIMD and In-place Processing. \\
			- Developed a simple printed character recognition algorithm based on NCC.	
			\item Developed defect detection algorithms for semi-conductor inspection machines.
			\item \underline{Tech Stack:} C++, SIMD, Image Processing, Computer Vision.

		\end{rSubsection}

		\begin{rSubsection}{Enscape}{Feb. 2017 - Oct. 2018}{Software Engineer}{South Korea}
			\item Developed defect inspection algorithms using Halcon library.
			\item Developed applications for semi-conductor inspection machines.
			\item \underline{Tech Stack:} C++, MFC, Halcon, OpenCV,  SIMD.

		\end{rSubsection}

		\begin{rSubsection}{Chowis}{Sept 2015 - Oct. 2016}{Software Engineer}{South Korea}
			\item Developed Android applications for a human skin analysis kit.
			\item \underline{Tech Stack:} C++, Java, Android\\ \\ \\ \\ 

		\end{rSubsection}
	\end{rSection}

	\begin{rSection}{Core Skills}
		\begin{tabular} {p{0.2\linewidth} p{0.8\linewidth}}
			\bf Programming & C++,  Python,  Pytorch, SIMD (SSE, AVX, NEON), TensorRT, OpenCV, Open3D, OpenGL, QT, MFC, CUDA, CMake \\
			\bf Research & Image Processing, Computer Vision,  Deep Learning,  Stereo Vision, Depth Estimation, 3D Gaussian Splatting, Pattern Matching, Object Detection, Machine Learning, 3D Rendering, Defect Inspection, Model Deployment \\
			\bf Language & Vietnamese, Korean, English
		\end{tabular}
	\end{rSection}		

	\begin{rSection}{Publication}\href{https://scholar.google.co.kr/citations?user=nhm8WBMAAAAJ&hl=en}{Google Scholar Profile}
	\begin{enumerate}[leftmargin=*]
		\item {\bf A Method for matching pattern using image and an apparatus of thereof},{HN. Doan, KS. Kwon},{S. Korea Domestic Patent},{2021},{\href{https://doi.org/10.8080/1020210025073}{No:1024319840000}}, \href{https://blog.naver.com/mvtech_ravid/222119961697}{View Demo}.
		\item {\bf Method for hole filling in 3D model, and recording medium and apparatus for performing the same},{ US Patent},{MC. Hong, BS. Kim, TD. Nguyen, HN. Doan},{2018},\href{https://patentimages.storage.googleapis.com/d5/ab/0a/111cb20d160a96/US9916694.pdf}{PDF}.
		\item {\bf Hole-Filling algorithm with spatio-temporal background information for view synthesis},{IEICE Trans. on Information and Systems},{HN. Doan, TD. Nguyen, MC. Hong},{2017},\href{https://www.jstage.jst.go.jp/article/transinf/E100.D/9/E100.D_2016PCP0010/_pdf}{PDF}.
		\item {\bf A spatial-temporal hole filling approach with background modeling and texture synthesis for 3D video},{ Proceedings of the 2015 Conf.  on research in adaptive and convergent systems},{HN. Doan, MC. Hong},{2015},\href{https://dl.acm.org/doi/abs/10.1145/2811411.2811497}{Link}.
		\item {\bf Hole filling algorithm using spatial-temporal background depth map for view synthesis in free view point television},{ Pacific Rim Conf.  on Multimedia},{HN. Doan, BS. Kim, MC. Hong},{2015},\href{https://link.springer.com/chapter/10.1007/978-3-319-24078-7_61}{Link}.
 		\item {\bf Directional hole filling algorithm in new view synthesis for 3D video using local segmentation},{ Proceedings of the 2014 Conf.  on Research in Adaptive and Convergent Systems},{HN. Doan, TA. Nguyen, MC. Hong},{2014},\href{https://dl.acm.org/doi/abs/10.1145/2663761.2664229}{Link}.
	\end{enumerate}	
	\end{rSection}

	\begin{rSection}{Personal Projects}
		{\bf \href{https://noidh.github.io/blog.html}{Technical Blog}}
		\\ I write many articles not only to share my knowledge about Image Processing, Computer Vision, Machine Learning, 3D Rendering and other miscellaneous but also to learn new technologies.  I also develop a small application to intuitively demonstrate how these algorithms work.

		{\bf  \href{https://noidh.github.io/pages/sw/ximagetool/ximagetool.html}{XImageTool}}
		\\ XImageTool is a free tool used for simulating fundamental Image Processing, Computer Vision, Machine Learning algorithms and 3D Rendering.

		{\bf \href{https://noidh.github.io/pages/sw/ximage2text/ximagetotext.html}{XText}}
		\\ XText is a free OCR software.
	\end{rSection}

\end{document}