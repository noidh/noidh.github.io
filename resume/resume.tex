\documentclass{resume}
\usepackage{xetexko}
\usepackage[utf8]{inputenc}
\usepackage[left=0.75in,top=0.6in,right=0.75in,bottom=0.6in]{geometry}
\usepackage{enumitem}
\usepackage{hyperref}
\hypersetup{
    colorlinks=true,
    linkcolor=blue,
    filecolor=magenta,      
    urlcolor=cyan,
    pdftitle={Doan Huu Noi},
    pdfpagemode=FullScreen,
    }

\urlstyle{same}

\name{DOAN HUU NOI}
\address{ +82-10-3674-3010 \\ doanhuunoi@gmail.com \\  \url{noidh.github.io}} 

\def\nameskip{\bigskip}
\def\sectionskip{\medskip}

\begin{document}

	\begin{rSection}{Education}
		{\bf Soongsil University, Seoul, South Korea} \hfill {\em Sept 2013 - July 2015} \\  {Master's Degree in Video \& Image Processing.} \smallskip \\
		{\bf Post and Telecom. Institute of Tech., HCM City, Vietnam} \hfill {\em Sept 2008 - Jan 2013} \\  {Bachelor's Degree in Computer Science.}
	\end{rSection}

	\begin{rSection}{Publication}\href{https://scholar.google.co.kr/citations?user=nhm8WBMAAAAJ&hl=en}{Google Scholar Profile}
	\begin{enumerate}[leftmargin=*]
		\item {\bf A Method for matching pattern using image and an apparatus of thereof},{HN. Doan, KS. Kwon},{S. Korea Domestic Patent},{2021},{\href{https://doi.org/10.8080/1020210025073}{No:1024319840000}}, \href{https://blog.naver.com/mvtech_ravid/222119961697}{View Demo}.
		\item {\bf Method for hole filling in 3D model, and recording medium and apparatus for performing the same},{ US Patent},{MC. Hong, BS. Kim, TD. Nguyen, HN. Doan},{2018},\href{https://patentimages.storage.googleapis.com/d5/ab/0a/111cb20d160a96/US9916694.pdf}{PDF}.
		\item {\bf Hole-Filling algorithm with spatio-temporal background information for view synthesis},{IEICE Trans. on Information and Systems},{HN. Doan, TD. Nguyen, MC. Hong},{2017},\href{https://www.jstage.jst.go.jp/article/transinf/E100.D/9/E100.D_2016PCP0010/_pdf}{PDF}.
		\item {\bf A spatial-temporal hole filling approach with background modeling and texture synthesis for 3D video},{ Proceedings of the 2015 Conf.  on research in adaptive and convergent systems},{HN. Doan, MC. Hong},{2015},\href{https://dl.acm.org/doi/abs/10.1145/2811411.2811497}{Link}.
		\item {\bf Hole filling algorithm using spatial-temporal background depth map for view synthesis in free view point television},{ Pacific Rim Conf.  on Multimedia},{HN. Doan, BS. Kim, MC. Hong},{2015},\href{https://link.springer.com/chapter/10.1007/978-3-319-24078-7_61}{Link}.
 		\item {\bf Directional hole filling algorithm in new view synthesis for 3D video using local segmentation},{ Proceedings of the 2014 Conf.  on Research in Adaptive and Convergent Systems},{HN. Doan, TA. Nguyen, MC. Hong},{2014},\href{https://dl.acm.org/doi/abs/10.1145/2663761.2664229}{Link}.
	\end{enumerate}	
\end{rSection}

	\begin{rSection}{Background}
		\begin{tabular} {p{0.2\linewidth} p{0.8\linewidth}}
			\bf Programming & C++,  SIMD (SSE, AVX), OpenGL, QT, MFC,CUDA \\
			\bf Research & Image Processing,  Computer Vision, Pattern Matching, Object Detection,  Machine Learning, 3D Rendering,Defect Inspection \\
			\bf Language & Vietnamese, Korean, English
		\end{tabular}
		
	\end{rSection}

	\begin{rSection}{Experience}
		\begin{rSubsection}{MVTech}{Nov. 2018 - Current}{Image Processing Researcher}{South Korea}
			\item Participating in developing RAVID framework \\
			- Developing the Shape Finder algorithm which is a feature-based matching algorithm. \href{https://blog.naver.com/mvtech_ravid/222119961697}{(View Demo)}\\ 			
- Developing very fast convolution and morphology operations by using SIMD and In-place Processing. \\
			- Developing an OCR algorithm based on NCC matching algorithm.\\	
			\item Developing defect inspection algorithms for automatic inspection machines.
			\item Technologies: C/C++, SIMD, Image Processing, Computer Vision.

		\end{rSubsection}

		\begin{rSubsection}{Enscape}{Feb. 2017 - Oct. 2018}{Software Engineer}{South Korea}
			\item Developing defect inspection algorithms by using Halcon library.
			\item Developing applications for Matrix machine, Auto-Handler machine.
			\item Technologies: C++, MFC, Halcon, OpenCV, SIMD.

		\end{rSubsection}

		\begin{rSubsection}{Chowis}{Sept 2015 - Oct. 2016}{Software Engineer}{South Korea}
			\item Developing Android application for a skin inspection kit.

		\end{rSubsection}
	\end{rSection}

	\begin{rSection}{Personal Projects}
		{\bf \href{https://noidh.github.io/blog.html}{Technical Blog}}
		\\ I write articles to describe what I have learnt about Image Processing, Computer Vision, Machine Learning, 3D Rendering and other miscellaneous.  Moreover,  I develop the XImageTool application to demonstrate how those algorithms work intuitively.

		{\bf  \href{https://noidh.github.io/pages/sw/ximagetool/ximagetool.html}{XImageTool}}
		\\ XImageTool is a free tool used for simulating fundamental Image Processing, Computer Vision, Machine Learning algorithms and 3D Rendering.

		{\bf \href{https://noidh.github.io/pages/sw/ximage2text/ximagetotext.html}{XImageToText}}
		\\ XImageToText is a free OCR software.

		{\bf \href{https://noidh.github.io/pages/sw/ximage_trainer/ximage_trainer.html}{XImageTrainer}}
		\\XImageTrainer is a free software used for preparing image data and collaborating with other DeepLearning frameworks to train the dataset. Without an efficient GUI tool, drawing bounding boxs for objects in thousands of images would be a painful job.  This application provides basic tools to make that task more convenient and simple.	
	\end{rSection}

\end{document}